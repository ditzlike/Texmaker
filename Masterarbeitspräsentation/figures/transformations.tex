    \begin{subfigure}[t]{0.3\textwidth}
        \centering
        \begin{tikzpicture}[
            remember picture,
            baseline=(vC.south),
            venn/.style={draw, circle, fill opacity=0.4, text opacity=1}
        ]
            \node[venn, minimum size=35pt] (vA) at (0,-1.9) {A};
            \node[venn, minimum size=22pt] (vB) at ($(vA.south west) + (0.1,-6.5pt)$) {B};
            \node[venn, minimum size=22pt] (vC) at ($(vA.south west) + (20pt,-7.5pt)$) {C};
            
            \node[venn, minimum size=35pt] (vY) at (-1.1,-1.5) {Y};
            \node[venn, minimum size=22pt] (vZ) at ($(vY.west) + (-3pt,0pt)$) {Z};
            \node[venn, minimum size=22pt] at ($(vY.north east) + (1.5pt,5pt)$) {X};
            
            \node[venn, minimum size=35pt] (vD) at (1.6,-1.5) {D};
            \node[venn, minimum size=22pt] at ($(vD.north west) + (-2.5pt,4pt)$) {E};
            \node[venn, minimum size=22pt] (vF) at ($(vA.north east) + (4pt,9pt)$) {F};

            \draw (0,-1.7) ellipse (2.5cm and 1.35cm);
            \node at ($(vZ |- vB)+(0,5pt)$) {R};
        \end{tikzpicture}
        \caption{Venn diagram of section overlap (\emph{\microv}).}
    \end{subfigure}
    \hfill
    \begin{subfigure}[t]{0.3\textwidth}
        \centering
        \begin{tikzpicture}[
            remember picture,
            baseline=(v2C.south),
            % venn/.style={draw=black!30, circle, fill opacity=0.4, text opacity=1}
            venn/.style={draw=none, circle, fill opacity=0.4, text opacity=1}
        ]
            \newcommand{\venn}[3]{ %{position}{size}{content}
                \node[venn, minimum size=#2] (v2#3) at #1 {};
                % \node[inner sep=2pt] (l2#3) at (v2#3) {#3};
                \node[inner sep=3pt] (l2#3) at (v2#3) {#3};
            }
        
            \venn{(0,-4.7)}{43pt}{A}
            \venn{($(v2A.south west) + (2pt,-3pt)$)}{14pt}{B}
            \venn{($(v2A.south west) + (22pt,-7pt)$)}{14pt}{C}
            \venn{(-1.4,-4.4)}{43pt}{Y}
            \venn{($(v2Y.west) + (-2pt,0)$)}{14pt}{Z}
            \venn{($(v2Y.north east) + (5pt,5pt)$)}{14}{X}
            \venn{(2.0,-4.3)}{45}{D}
            \venn{($(v2D.north west) + (0,4pt)$)}{14}{E}
            \venn{($(v2A.north east) + (8pt,9pt)$)}{14}{F}
            
            \draw[black!60, ultra thick] 
                (l2A)--(l2B) (l2A)--(l2C) 
                (l2Y)--(l2X) (l2Y)--(l2Z)
                (l2F)--(l2E) (l2E)--(l2D);
            \draw[black!25, very thick] (l2A)--(l2Y) (l2A)--(l2F);
            
            % cell border
            \draw[
                every node/.style={sloped,anchor=south,near end},
                ultra thick, dashed, red
            ]
                ($(v2A.north)+(0,-0.3cm)$) -- +(south west:1.8cm)
                ($(v2A.north)+(0,-0.3cm)$) -- +(south east:1.8cm)
                ($(v2A.north)+(0,-0.3cm)$) -- +(north:0.9cm);
        \end{tikzpicture}
        \caption{Abstracted \emph{network view}, with sections partitioned into cells.
        % Ralf: Wenn wir die nächste Zeile weglassen, sparen wir eine Zeile Platz
        %(separated by dashed lines)
        }
    \end{subfigure}
    \hfill
    \begin{subfigure}[t]{0.3\textwidth}
        \centering
        \begin{tikzpicture}[
            remember picture,
            baseline=(a.south)
        ]

            \node[circle,draw,very thick,fill=myblue!70,inner sep=9pt] (a) at (0,0) {A};
            \node[circle,draw,very thick,fill=mygreen!70,inner sep=5pt] (y) at (-0.8,1.3) {Y};
            \node[circle,draw,very thick,fill=myred!60,inner sep=5pt] (d) at (0.8,1.3) {D};
            \draw (a)--(d);
            \draw[line width=3pt] (a)--(y);
            
        \end{tikzpicture}
        \caption{Interpretable \emph{\macrov}.}
    \end{subfigure}
    %
    % inter-picture connections
    \begin{tikzpicture}[
        overlay,
        remember picture,
        transformation/.style={black!75,dotted,-{Latex[scale=2]}},
    ]
        \draw[transformation,bend angle=20,bend right] ($(v2A)+(6pt,-4pt)$) to (a);
        \draw[transformation,bend angle=22,bend left] ($(v2Y)+(6pt,3pt)$) to (y);
        \draw[transformation,bend angle=20,bend right] ($(v2D)+(6pt,-3pt)$) to (d);
    \end{tikzpicture}

